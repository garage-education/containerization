%----------------------------------------------------------------------------------------------------------------------%


\section{Docker - Long-lived containers}\label{sec:docker-long-lived-containers}
%----------------------------------------------------------------------------------------------------------------------%

\subsection{docker detached}\label{subsec:docker-detached}
\begin{frame}{docker detached}
    \begin{itemize}
        \item \texttt{-d, --detach} Run container in background and print container ID.
        \pause
        \lstinputlisting[language=bash, linerange={215-220}]{docker/shell-samples.sh}
    \end{itemize}
\end{frame}

\subsection{docker expose container port}\label{subsec:docker-expose-container-port}
\begin{frame}{docker expose container port}
    \begin{itemize}
        \item \texttt{port} List port mappings or a specific mapping for the container.
        \pause
        \item \texttt{-p, --publish list} Publish a container's port(s) to the host.
        \pause
        \lstinputlisting[language=bash, linerange={222-229}]{docker/shell-samples.sh}
    \end{itemize}
\end{frame}
\begin{frame}{docker bridge networking}
    \begin{figure}[!t]
        \raggedright
        \input{docker/figures/docker-bridge-httpd.tikz}
        \caption{Docker Bridge Networking}
    \end{figure}
\end{frame}

\subsection{docker show container logs}\label{subsec:docker-show-container-logs}
\begin{frame}{docker show container logs}
    \begin{itemize}
        \item \texttt{logs} Fetch the logs of a container.
        \pause
        \item \texttt{-t, --timestamps} Show timestamps.
        \pause
        \item \texttt{-f, --follow} Follow log output.
        \pause
        \lstinputlisting[language=bash, linerange={231-239}]{docker/shell-samples.sh}
    \end{itemize}
\end{frame}

\subsection{docker restart container}\label{subsec:docker-restart-container}
\begin{frame}{docker restart}
    \begin{itemize}
        \item \texttt{restart} Restart one or more containers.
        \pause
        \item \texttt{-t, --time int} Seconds to wait for stop before killing the container (default 10).
        \pause
        \lstinputlisting[language=bash, linerange={241-249}]{docker/shell-samples.sh}
    \end{itemize}
\end{frame}

\subsection{docker stop container}\label{subsec:docker-stop-container}
\begin{frame}{docker stop container}
    \begin{itemize}
        \item \texttt{stop} Stop one or more running containers.
        \pause
        \item \texttt{-t, --time int} Seconds to wait for stop before killing it (default 10).
        \pause
        \lstinputlisting[language=bash, linerange={250-260}]{docker/shell-samples.sh}
    \end{itemize}
\end{frame}

\subsection{docker kill container}\label{subsec:docker-kill-container}
\begin{frame}{docker kill container}
    \begin{itemize}
        \item \texttt{Kill} Kill one or more running containers.
        \pause
        \lstinputlisting[language=bash, linerange={262-265}]{docker/shell-samples.sh}
    \end{itemize}
\end{frame}

\subsection{docker stats}\label{subsec:docker-stats}
\begin{frame}{docker stats}
    \begin{itemize}
        \item \texttt{stats} Display a live stream of container(s) resource usage statistics.
        \pause
        \item \texttt{-a, --all} Show all containers (default shows just running).
        \pause
        \item \texttt{--no-stream} Disable streaming stats and only pull the first result.
        \pause
        \lstinputlisting[language=bash, linerange={267-274}]{docker/shell-samples.sh}
    \end{itemize}
\end{frame}

\subsection{docker top}\label{subsec:docker-top}
\begin{frame}{docker top}
    \begin{itemize}
        \item \texttt{top} Display the running processes of a container.
        \pause
        \lstinputlisting[language=bash, linerange={276-281}]{docker/shell-samples.sh}
    \end{itemize}
\end{frame}

\subsection{docker pause and unpause}\label{subsec:docker-pause-and-unpause}
\begin{frame}{docker pause and unpause}
    \begin{itemize}
        \item \texttt{pause} Pause all processes within one or more containers.
        \pause
        \item \texttt{unpause} Unpause all processes within one or more containers.
        \pause
        \lstinputlisting[language=bash, linerange={283-294}]{docker/shell-samples.sh}
    \end{itemize}
\end{frame}

\subsection{docker exec}\label{subsec:docker-exec}
\begin{frame}{docker exec}
    \begin{itemize}
        \item \texttt{exec} Run a command in a running container.
        \pause
        \item \texttt{-i, --interactive} Keep STDIN open even if not attached.
        \pause
        \item \texttt{-t, --tty} Allocate a pseudo-TTY.
        \pause
        \lstinputlisting[language=bash, linerange={296-300,302-305,307-309}]{docker/shell-samples.sh}
    \end{itemize}
\end{frame}

\subsection{docker cp}\label{subsec:docker-cp}
\begin{frame}{docker cp}
    \begin{itemize}
        \item \texttt{cp} Copy files/folders between a container and the local filesystem.
        \pause
        \lstinputlisting[language=bash, linerange={317-319}]{docker/shell-samples.sh}
    \end{itemize}
\end{frame}

\subsection{docker wait}\label{subsec:docker-wait}
\begin{frame}{docker wait}
    \begin{itemize}
        \item \texttt{wait} Block until one or more containers stop, then print their exit codes.
        \pause
        \lstinputlisting[language=bash, linerange={321-324}]{docker/shell-samples.sh}
    \end{itemize}
\end{frame}

\subsection{docker attach}\label{subsec:docker-attach}
\begin{frame}{docker attach}
    \begin{itemize}
        \item \texttt{attach} Attach local standard input, output, and error streams to a running container.
        \pause
        \lstinputlisting[language=bash, linerange={326-326}]{docker/shell-samples.sh}
    \end{itemize}
\end{frame}
%----------------------------------------------------------------------------------------------------------------------%