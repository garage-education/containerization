%----------------------------------------------------------------------------------------------------------------------%

\section{Introduction - Docker Overview}\label{sec:introduction-docker-overview}
%----------------------------------------------------------------------------------------------------------------------%

\subsection{What is docker?}\label{subsec:what-is-docker}
\begin{frame}{What is docker?}
    \begin{itemize}[<+- | alert@+>]
        \item Docker is an OS-level virtualization tool.
        \item Docker is an open platform for developing, shipping, and running applications.
        \item Docker provides tools, and a platform to manage the lifecycle of your containers:
        \begin{itemize}
            \item Develop your application and its supporting components using containers.
            \item The container becomes the unit for distributing and testing your application.
            \item When you are ready, deploy your application into your production environment, as a container or an orchestrated service.
            \item This works the same whether your production environment is a local data center, a cloud provider, or a hybrid of the two.
        \end{itemize}
    \end{itemize}
\end{frame}

\subsection{Docker Architecture}\label{subsec:docker-architecture}
\begin{frame}{Docker Architecture}
    \begin{itemize}
        \item Docker uses a client-server architecture.
        \linebreak
        \pause
        \begin{figure}[!t]
            \raggedright
            \begin{tikzpicture}[
    node/.style = {
        scale=0.70,
        node distance = 2mm,
        rectangle,
        draw = black,
        thick,
        text centered,
        minimum height = 10mm,
        minimum width = 10mm
    },
    txt/.style = {
        scale=0.70,
        node distance = 2mm,
        rectangle,
        text centered,
        minimum height = 2mm,
        minimum width = 10mm
    }
]
    \tikzstyle{layer} = [inner sep=1mm, draw=black, thick, fill= gray!30]
    \tikzstyle{vm} = [fill = gray!30]
    \tikzstyle{hw} = [fill = blue!35]
    \tikzstyle{sw} = [fill = blue!15]
    \tikzstyle{hv} = [fill = blue!5 ]
    \tikzstyle{ar} = [fill = black  ]

    %%%%%%%%%%%%%%%%%
    % Docker Client %
    %%%%%%%%%%%%%%%%%
    \node [txt, text width=3cm, visible on=<1->] (CLIENT) {Client};
    \node [node, hw, above = of CLIENT, text width=3cm, visible on=<3->] (REST) {Rest};
    \node [node, hw, above = of REST, text width=3cm, visible on=<4->] (CLI) {CLI};
    \begin{scope}[on background layer]
        \node [fit = (CLIENT) (REST) (CLI), layer, visible on=<1->] {};
    \end{scope}

    %%%%%%%%%%%%%%%%%
    % Docker Server %
    %%%%%%%%%%%%%%%%%
    \node [txt, right = of CLIENT, text width=4cm, xshift = 2mm, visible on=<5->] (SER) {Server - Docker Demon};
    \node [node, hw, above = of SER, text width=4cm, visible on=<6->] (IMG) {Images};
    \node [node, hw, above = of IMG, text width=4cm, visible on=<7->] (CON) {Containers};
    \node [node, hw, above = of CON, text width=4cm, visible on=<8->] (VOL) {Volumes};
    \begin{scope}[on background layer]
        \node [fit = (SER) (CON) (IMG) (VOL), layer, visible on=<5->] {};
    \end{scope}

    %%%%%%%%%%%%%%%%%%%
    % Docker Rigestry %
    %%%%%%%%%%%%%%%%%%%
    \node [txt, right = of SER, text width=3cm, xshift = 2mm, visible on=<9->] (REG) {Registry};
    \node [node, hw, above = of REG, text width=3cm, visible on=<10->] (HUB) {Docker Hub};
    \begin{scope}[on background layer]
        \node [fit = (REG) (HUB), layer, visible on=<9->] {};
    \end{scope}
\end{tikzpicture}
            \caption{Docker Architecture}
        \end{figure}
    \end{itemize}
\end{frame}
\begin{frame}{Docker Demon (Server)}
    \begin{itemize}[<+- | alert@+>]
        \item The Docker daemon \texttt{(dockerd)} listens for Docker API requests and manages Docker objects.
        \item A daemon can also communicate with other daemons to manage Docker services.

    \end{itemize}
\end{frame}
\begin{frame}{Docker Objects}
    \begin{itemize}[<+- | alert@+>]
        \item \textbf{Images} are a read-only template with instructions for creating a Docker container.
        \item \textbf{Containers} are a runnable instance of an image.
        \item \textbf{Volumes} are the preferred mechanism for persisting data generated by and used by Docker containers.
    \end{itemize}
\end{frame}
\begin{frame}{Docker Client}
    \begin{itemize}[<+- | alert@+>]
        \item The Docker client \texttt{(docker)} is the primary way that many Docker users interact with Docker.
        \item When you use commands such as \texttt{(docker run)}, the client sends these commands to \texttt{(dockerd)}, which carries them out.
        \item The docker command uses the Docker API and can communicate with one or more docker daemons.
    \end{itemize}
\end{frame}
\begin{frame}{Docker Registry}
    \begin{itemize}[<+- | alert@+>]
        \item A Docker registry stores Docker images.
        \item Docker Hub is a public registry that anyone can use, and by default Docker configurations looks for the images on Docker Hub.
        \item Docker Hub is not the only registry in the market, and you can use your own docker registry.
    \end{itemize}
\end{frame}
%----------------------------------------------------------------------------------------------------------------------%