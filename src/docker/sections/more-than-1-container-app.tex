%----------------------------------------------------------------------------------------------------------------------%


\section{Docker - More than one container app}\label{sec:more-than-one-container-app}
%----------------------------------------------------------------------------------------------------------------------%

\subsection{Docker default bridge}\label{subsec:docker-default-bridge}
\begin{frame}{Docker default bridge}
    \begin{itemize}
        \item What if we have an application with more than one container.
        \item Ex; WordPress rich content management system uses apache httpd and mysql servers.
        \item \texttt{-e, --env list} Set environment variables.
        \lstinputlisting[language=bash, linerange={329-334}]{docker/shell-samples.sh}
    \end{itemize}
\end{frame}

\subsection{User-defined bridge}\label{subsec:user-defined-bridge}
\begin{frame}{User-defined bridge}
    IMAGE GOES HERE
\end{frame}
\begin{frame}{User-defined bridge}
    \begin{itemize}
        \item User-defined bridges provide automatic DNS resolution between containers.
        \item User-defined bridges provide better isolation.
        \item Containers can be attached and detached from user-defined networks on the fly.
        \item Each user-defined network creates a configurable bridge.
        \item Linked containers on the default bridge network share environment variables.
    \end{itemize}
\end{frame}

\subsection{Docker Network Commands}\label{subsec:docker-network-commands}
\begin{frame}{Docker Network Commands}
    \begin{itemize}
        \item \texttt{create} Create a network.
        \item \texttt{ls} List networks.
        \item \texttt{connect} Connect a container to a network.
        \item \texttt{inspect} Display detailed information on one or more networks.
        \item \texttt{rm} Deletes one or more networks.
        \lstinputlisting[language=bash, linerange={336-344}]{docker/shell-samples.sh}
    \end{itemize}
\end{frame}