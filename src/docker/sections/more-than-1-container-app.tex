%----------------------------------------------------------------------------------------------------------------------%

\section{Docker - More than one container app}\label{sec:more-than-one-container-app}
%----------------------------------------------------------------------------------------------------------------------%

\subsection{Docker default bridge}\label{subsec:docker-default-bridge}
\begin{frame}{Docker default bridge}
    \begin{itemize}
        \item What if we have an application with more than one container.
        \pause
        \item Ex; WordPress rich content management system uses apache httpd and mysql servers.
        \pause
        \item \texttt{-e, --env list} Set environment variables.
        \pause
        \lstinputlisting[language=bash, linerange={329-334}]{docker/shell-samples.sh}
    \end{itemize}
\end{frame}

\subsection{User-defined bridge}\label{subsec:user-defined-bridge}
\begin{frame}{User-defined bridge}
    \begin{figure}[!t]
        \raggedright
        \begin{tikzpicture}[
    node/.style = {
        scale=0.70,
        node distance = 2mm,
        rectangle,
        draw = black,
        thick,
        text centered,
        minimum height = 10mm,
        minimum width = 10mm
    },
    txt/.style = {
        scale=0.70,
        node distance = 2mm,
        rectangle,
        text centered,
        minimum height = 2mm,
        minimum width = 10mm
    }
]
    \tikzstyle{layer} = [inner sep=1mm, draw=black, thick, fill= gray!30]
    \tikzstyle{de} = [inner sep=1mm, draw=black, thick, fill= gray!60]
    \tikzstyle{cr} = [inner sep=1mm, draw=black, thick, fill= gray!90]
    \tikzstyle{dn} = [inner sep=1mm, draw=black, thick, opacity=0.6, fill= white!40]
    \tikzstyle{vm} = [fill = gray!30]
    \tikzstyle{hw} = [fill = blue!35]
    \tikzstyle{sw} = [fill = blue!15]
    \tikzstyle{hv} = [fill = blue!5 ]
    \tikzstyle{ar} = [fill = black  ]

    \node [txt, text width=8cm, visible on=<1->] (OS) {Operating System};
    \node [node, hw, above = of OS, text width=8cm, visible on=<2->] (PN) {Physical Network};
    \node [txt, above = of PN, text width=7.7cm, visible on=<3->] (DE) {Docker Engine};
    \node [txt, above = of DE, text width=4.0cm, visible on=<7->] (UD) {User-defined Network};
    % Container 1 structure
    \node [node, hw, above = of UD, xshift = -2.0cm, text width=2.8cm, visible on=<5->] (P1) {Bridge};
    \node [node, fill= gray!90, above = of P1, text width=2.8cm, visible on=<6->] (I1) {10.0.0.2};
    \node [txt, above = of I1, text width=3.3cm, visible on=<4->] (C1) {Container};
    % Container 2 structure
    \node [node, hw, right = of P1, xshift = 6.5mm, text width=2.8cm, visible on=<5->] (P2) {Bridge};
    \node [node, fill= gray!90, above = of P2, text width=2.8cm, visible on=<6->] (I2) {10.0.0.3};
    \node [txt, above = of I2, text width=3.3cm, visible on=<4->] (C2) {Container};
    % Empty space
    \node [txt, above = of C1, text width=3.3cm, visible on=<1->] (T1) {};
    \node [txt, above = of T1, text width=3.3cm, visible on=<1->] (T2) {};
    % Operating system group
    \begin{scope}[on background layer]
        \node [fit = (OS) (PN) (DE) (P1) (I1) (C1) (P2) (I2) (C2) (T1) (T2), layer, visible on=<1->] {};
    \end{scope}
    % Docker Engine group
    \begin{scope}[on background layer]
        \node [fit = (DE) (P1) (I1) (C1) (P2) (I2) (C2) (T1), de, visible on=<3->] {};
    \end{scope}
    % Container 1 group
    \begin{scope}[on background layer]
        \node [fit = (P1) (I1) (C1), cr, visible on=<4->] {};
    \end{scope}
    % Container 2 group
    \begin{scope}[on background layer]
        \node [fit = (P2) (I2) (C2), cr, visible on=<4->] {};
    \end{scope}
    % User-defined bridge networking
    \begin{scope}[on background layer]
        \node [fit = (UD) (P1) (I1) (I2) (P2), dn, visible on=<7->] {};
    \end{scope}
\end{tikzpicture}
        \caption{User-defined Bridge Networking}
    \end{figure}
\end{frame}
\begin{frame}{User-defined bridge}
    \begin{itemize}
        \item User-defined bridges provide automatic DNS resolution between containers.
        \pause
        \item User-defined bridges provide better isolation.
        \pause
        \item Containers can be attached and detached from user-defined networks on the fly.
        \pause
        \item Each user-defined network creates a configurable bridge.
        \pause
        \item Linked containers on the default bridge network share environment variables.
    \end{itemize}
\end{frame}

\subsection{Docker Network Commands}\label{subsec:docker-network-commands}
\begin{frame}{Docker Network Commands}
    \begin{itemize}
        \item \texttt{create} Create a network.
        \pause
        \item \texttt{ls} List networks.
        \pause
        \item \texttt{connect} Connect a container to a network.
        \pause
        \item \texttt{inspect} Display detailed information on one or more networks.
        \pause
        \item \texttt{rm} Deletes one or more networks.
        \pause
        \lstinputlisting[language=bash, linerange={336-344}]{docker/shell-samples.sh}
    \end{itemize}
\end{frame}