%----------------------------------------------------------------------------------------------------------------------%

\section{Containerization}\label{sec:containerization}
%----------------------------------------------------------------------------------------------------------------------%

\subsection{Containers}\label{subsec:containers}
\begin{frame}{What is a Container?}
    \begin{itemize}[<+- | alert@+>]
        \item The process of virtualizing the operating system produces containers.
        \item Container is a virtual operating system.
        \item A container is an abstraction at the OS layer that packages code and dependencies together as a standardized unit of software.
        \item Containers take up less space than VMs, boots quickly, and in isolation.
        \item Containerization eliminates infrastructure wasted resources and utilizes them.
    \end{itemize}
\end{frame}

\subsection{Container Deployment}\label{subsec:container-deployment}
\begin{frame}{Container Deployment}
    \begin{figure}[!t]
        \raggedright
        \begin{tikzpicture}[
    node/.style = {
        scale=0.40,
        node distance = 2mm,
        rectangle,
        draw = black,
        thick,
        text centered,
        minimum height = 10mm,
        minimum width = 10mm
    },
    txt/.style = {
        scale = 0.40,
        node distance = 2mm,
        rectangle,
        text centered,
        minimum height = 2mm,
        minimum width = 10mm
    }
]
    \tikzstyle{layer} = [inner sep=1mm, draw=black, thick, fill= gray!30]
    \tikzstyle{vm} = [fill = gray!30]
    \tikzstyle{hw} = [fill = blue!35]
    \tikzstyle{sw} = [fill = blue!15]
    \tikzstyle{hv} = [fill = blue!5 ]
    \tikzstyle{ar} = [fill = black  ]

    %%%%%%%%%%%%%%%%%%%%%%%%%%
    % Traditional Deployment %
    %%%%%%%%%%%%%%%%%%%%%%%%%%
    \node [node, sw, text width=1.82cm, visible on=<1->] (A11) {application};
    \node [node, sw, right = of A11, text width=1.82cm, visible on=<1->] (A12) {application};
    \node [node, sw, right = of A12, text width=1.82cm, visible on=<1->] (A13) {application};
    \node [node, hw, below = of A12, text width=7cm, visible on=<1->] (OS1) {Operating System};
    \node [node, hw, below = of OS1, text width=7cm, visible on=<1->] (HW1) {Hardware};

    %%%%%%%%%
    % ARROW %
    %%%%%%%%%
    \node [node, ar, right = of OS1, text width=1cm, visible on=<1->, single arrow] (AR1) {};

    %%%%%%%%%%%%%%%%%%%%%%%%%%%%%
    % Virtualization Deployment %
    %%%%%%%%%%%%%%%%%%%%%%%%%%%%%
    \node [node, hw, right = of AR1, text width=7cm, visible on=<1->] (OS2) {Operating System};
    \node [node, hw, below = of OS2, text width=7cm, visible on=<1->] (HW2) {Hardware};
    \node [node, hv, above = of OS2, text width=7cm, visible on=<1->] (HV1) {Hypervisor};
    % Virtual Machine 1
    \node [txt , vm, above = of HV1, text width=2.75cm, xshift = -1.8cm, visible on=<1->] (VM1) {Virtual Machine};
    \node [node, hw, above = of VM1, text width=2.75cm, yshift=-2mm, visible on=<1->] (OS3) {OS};
    \node [node, hw, above = of OS3, text width=2.75cm, visible on=<1->] (BL1) {Bin/Lib};
    \node [node, sw, above = of BL1, text width=0.96cm, xshift = -0.9cm, visible on=<1->] (A21) {app};
    \node [node, sw, right = of A21, text width=0.96cm, visible on=<1->] (A22) {app};
    \begin{scope}[on background layer]
        \node [fit = (VM1) (OS3) (BL1) (A21) (A22), layer, visible on=<1->] {};
    \end{scope}
    % Virtual Machine 2
    \node [txt , right = of VM1, text width=2.75cm, xshift = 1mm, visible on=<1->] (VM2) {Virtual Machine};
    \node [node, hw, above = of VM2, text width=2.75cm, yshift=-2mm, visible on=<1->] (OS4) {OS};
    \node [node, hw, above = of OS4, text width=2.75cm, visible on=<1->] (BL2) {Bin/Lib};
    \node [node, sw, above = of BL2, text width=0.96cm, xshift = -0.9cm, visible on=<1->] (A23) {app};
    \node [node, sw, right = of A23, text width=0.96cm, visible on=<1->] (A24) {app};
    \begin{scope}[on background layer]
        \node [fit = (VM2) (OS4) (BL2) (A23) (A24), layer, visible on=<1->] {};
    \end{scope}

    %%%%%%%%%
    % ARROW %
    %%%%%%%%%
    \node [node, ar, right = of OS2, text width=1cm, visible on=<1->, single arrow] (AR2) {};

    %%%%%%%%%%%%%%%%%%%%%%%%%%%%%%%
    % Containerization Deployment %
    %%%%%%%%%%%%%%%%%%%%%%%%%%%%%%%
    \node [node, hw, right = of AR2, text width=7cm, visible on=<3->] (OS3) {Operating System};
    \node [node, hw, below = of OS3, text width=7cm, visible on=<2->] (HW3) {Hardware};
    \node [node, hv, above = of OS3, text width=7cm, visible on=<4->] (HV2) {Container Runtime};
    % Container 1
    \node [txt , vm, above = of HV2, text width=1.58cm, xshift = -2.45cm, visible on=<5->] (VM3) {Container};
    \node [node, hw, above = of VM3, text width=1.58cm, yshift=-2mm, visible on=<6->] (BL3) {Bin/Lib};
    \node [node, sw, above = of BL3, text width=1.58cm, visible on=<7->] (A31) {app};
    \begin{scope}[on background layer]
        \node [fit = (VM3) (BL3) (A31), layer, visible on=<5->] {};
    \end{scope}
    % Container 2
    \node [txt , vm, right = of VM3, text width=1.58cm, xshift = 1mm, visible on=<8->] (VM4) {Container};
    \node [node, hw, above = of VM4, text width=1.58cm, yshift = -2mm, visible on=<8->] (BL4) {Bin/Lib};
    \node [node, sw, above = of BL4, text width=1.58cm, visible on=<8->] (A32) {app};
    \begin{scope}[on background layer]
        \node [fit = (VM4) (BL4) (A32), layer, visible on=<8->] {};
    \end{scope}
    % Container 3
    \node [txt , vm, right = of VM4, text width=1.58cm, xshift = 1mm, visible on=<9->] (VM5) {Container};
    \node [node, hw, above = of VM5, text width=1.58cm, yshift = -2mm, visible on=<9->] (BL5) {Bin/Lib};
    \node [node, sw, above = of BL5, text width=1.58cm, visible on=<9->] (A33) {app};
    \begin{scope}[on background layer]
        \node [fit = (VM5) (BL5) (A33), layer, visible on=<9->] {};
    \end{scope}
\end{tikzpicture}
        \caption{Containerization Deployment}
    \end{figure}
\end{frame}

\subsection{OS-level virtualization}\label{subsec:os-level-virtualization}
\begin{frame}{OS-level virtualization}
    \begin{itemize}[<+- | alert@+>]
        \item How is containerization possible?
        \begin{itemize}
            \item OS-level virtualization refers to an operating system paradigm in which the operating system allows the existence of multiple isolated user-space instances (containers).
            \item OS-level virtualization solutions are the container engines.
            \item A container engine is a managed environment for deploying containerized applications.
        \end{itemize}
        \item User-space instances have different names
        \begin{itemize}
            \item Containers in \textbf{Docker} and Linux containers \textbf{LXC}.
            \item VPS in \textbf{OpenVZ}
            \item Virtual Kernel \textbf{DragonFly BSD}
        \end{itemize}
    \end{itemize}
\end{frame}
%----------------------------------------------------------------------------------------------------------------------%