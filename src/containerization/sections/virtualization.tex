%----------------------------------------------------------------------------------------------------------------------%

\section{Virtualization}\label{sec:virtualization}
%----------------------------------------------------------------------------------------------------------------------%

\subsection{Virtual Machines}\label{subsec:virtual-machines}
\begin{frame}{Virtual Machines}
    \begin{itemize}[<+- | alert@+>]
        \item Virtualizing hardware produces virtual machines (VMs).
        \item Virtualization allows you to run multiple VMs on a single physical server.
        Each VM includes a full copy of an operating system, the application, necessary binaries, and libraries - taking up tens of GBs.
        \item Virtualization allows more effortless adding and updating of applications that solve the scalability issue.
        \item Virtualization allows better utilization of resources.
        \item Virtualization isolates applications between VMs.
    \end{itemize}
\end{frame}

\subsection{Virtualized Deployment}\label{subsec:virtualized-deployment}
\begin{frame}{Virtualized Deployment}
    \begin{figure}[!t]
        \raggedright
        \begin{tikzpicture}[
    node/.style={
        scale=0.55,
        node distance = 2mm,
        rectangle,
        draw = black,
        thick,
        text centered,
        minimum height = 10mm,
        minimum width = 10mm
    },
    txt/.style = {
        scale = 0.55,
        node distance = 2mm,
        rectangle,
        text centered,
        minimum height = 2mm,
        minimum width = 10mm
    }
]
    \tikzstyle{layer} = [inner sep=1mm, draw=black, thick, fill= gray!30]
    \tikzstyle{vm} = [fill = gray!30]
    \tikzstyle{hw} = [fill = blue!35]
    \tikzstyle{sw} = [fill = blue!15]
    \tikzstyle{hv} = [fill = blue!5 ]
    \tikzstyle{ar} = [fill = black  ]

    \node [node, sw, text width=1.90cm, visible on=<1->] (A11) {application};
    \node [node, sw, right = of A11, text width=1.90cm, visible on=<1->] (A12) {application};
    \node [node, sw, right = of A12, text width=1.90cm, visible on=<1->] (A13) {application};
    \node [node, hw, below = of A12, text width=7cm, visible on=<1->] (OS1) {Operating System};
    \node [node, hw, below = of OS1, text width=7cm, visible on=<1->] (HW1) {Hardware};

    \node [node, ar, right = of OS1, text width=1cm, visible on=<2->, single arrow] (AR1) {};

    \node [node, hw, right = of AR1, text width=7cm, visible on=<4->] (OS2) {Operating System};
    \node [node, hw, below = of OS2, text width=7cm, visible on=<3->] (HW2) {Hardware};
    \node [node, hv, above = of OS2, text width=7cm, visible on=<5->] (HV1) {Hypervisor};
    % Virtual Machine 2
    \node [txt , vm, above = of HV1, text width=2.92cm, xshift = -1.8cm, visible on=<6->] (VM1) {Virtual Machine};
    \node [node, hw, above = of VM1, text width=2.92cm, yshift=-2mm, visible on=<7->] (OS3) {OS};
    \node [node, hw, above = of OS3, text width=2.92cm, visible on=<8->] (BL1) {Bin/Lib};
    \node [node, sw, above = of BL1, text width=1.13cm, xshift = -0.9cm, visible on=<9->] (A21) {app};
    \node [node, sw, right = of A21, text width=1.13cm, visible on=<10->] (A22) {app};
    % Virtual Machine 2
    \node [txt , right = of VM1, text width=2.92cm, xshift = 1mm, visible on=<11->] (VM2) {Virtual Machine};
    \node [node, hw, above = of VM2, text width=2.92cm, yshift=-2mm, visible on=<11->] (OS4) {OS};
    \node [node, hw, above = of OS4, text width=2.92cm, visible on=<11->] (BL2) {Bin/Lib};
    \node [node, sw, above = of BL2, text width=1.13cm, xshift = -0.9cm, visible on=<11->] (A23) {app};
    \node [node, sw, right = of A23, text width=1.13cm, visible on=<11->] (A24) {app};

    \begin{scope}[on background layer]
        \node [fit = (VM1) (OS3) (BL1) (A21) (A22), layer, visible on=<6->] {};
    \end{scope}
    \begin{scope}[on background layer]
        \node [fit = (VM2) (OS4) (BL2) (A23) (A24), layer, visible on=<11->] {};
    \end{scope}
\end{tikzpicture}
        \caption{Virtualization Deployment}
    \end{figure}
\end{frame}

\subsection{Hypervisor}\label{subsec:hypervisor}
\begin{frame}{Hypervisor}
    \begin{itemize}[<+- | alert@+>]
        \item How is virtualization possible?
        \begin{itemize}
            \item A hypervisor is computer software, firmware or hardware that creates and runs virtual machines.
            \item The hypervisor allows multiple VMs to run on a single machine.
        \end{itemize}
        \item The hypervisor has 2 types:
    \end{itemize}
\end{frame}

\subsection{Type 1 Hypervisor}\label{subsec:type-1-hypervisor}
\begin{frame}{Type 1 Hypervisor}
    Type-1, native, or bare-metal hypervisors.
    \pause
    \metroset{block=fill}
    \begin{block}{Examples}
        \begin{itemize}
            \item Type-1: VMware ESX and Citrix Xen servers.
        \end{itemize}
    \end{block}
    \pause
    \begin{figure}[!t]
        \raggedright
        \begin{tikzpicture}[
    node/.style={
        scale=0.75,
        node distance=2mm,
        rectangle,
        draw=black,
        thick,
        text centered,
        minimum height=10mm,
        minimum width=10mm
    }
]
    \tikzstyle{hw} = [fill = blue!35]
    \tikzstyle{sw} = [fill = blue!15]
    \tikzstyle{hv} = [fill = blue!5 ]

    \node [node, sw, text width=1.95cm, visible on=<5->] (A1) {Guest OS};
    \node [node, sw, right = of A1, text width=1.95cm, visible on=<6->] (A2) {Guest OS};
    \node [node, sw, right = of A2, text width=1.95cm, visible on=<7->] (A3) {Guest OS};
    \node [node, hv, below = of A2, text width=7cm, visible on=<4->] (HV) {Hypervisor};
    \node [node, hw, below = of HV, text width=7cm, visible on=<3->] (HW) {Hardware};
\end{tikzpicture}
        \caption{Type 1, native, or bare-metal hypervisors}
    \end{figure}
\end{frame}

\subsection{Type 2 Hypervisor}\label{subsec:type-2-hypervisor}
\begin{frame}{Type 2 Hypervisor}
    Type-2, or hosted hypervisors.
    \pause
    \metroset{block=fill}
    \begin{block}{Examples}
        \begin{itemize}
            \item Type-2: VMware player and VirtualBox.
        \end{itemize}
    \end{block}
    \pause
    \begin{figure}[!t]
        \raggedright
        \input{containerization/figures/type-2-hypervisor.tikz}
        \caption{Type 2, or Hosted Hypervisor}
    \end{figure}
\end{frame}
%----------------------------------------------------------------------------------------------------------------------%