%----------------------------------------------------------------------------------------------------------------------%

\section{Summary and Popular Questions}\label{sec:popular-questions}
%----------------------------------------------------------------------------------------------------------------------%

\subsection{Summary}\label{subsec:summary}
\begin{frame}
    \begin{itemize}[<+- | alert@+>]
        \item The word virtualization applies to hardware and operating system.
        \item Hardware virtualization produces virtual machines.
        \item Operating system virtualization produces containers.
    \end{itemize}
\end{frame}

\subsection{Why Containers are more popular?}\label{subsec:containers-are-more-popular}
\begin{frame}{Why Containers are more popular?}
    \begin{itemize}[<+- | alert@+>]
        \item Containerization gives us better resource isolation with predictable application performance.
        \item Containerization gives us better resource utilization with high efficiency and density.
        \item They are loosely coupled, distributed, elastic, liberated micro-services.
        \item Environmental consistency across development, testing, and production \textbf{"It worked on my machine."}
        \item Agile application creation and deployment.
    \end{itemize}
\end{frame}

\subsection{What is hybrid container architecture?}\label{subsec:what-is-hybrid-container-architecture?}
\begin{frame}{What is hybrid container architecture?}
    \begin{itemize}[<+- | alert@+>]
        \item A hybrid container architecture is an architecture combining virtualization on both hardware and OS levels.
        \item Example: The container engine and associated containers execute on top of a virtual machine.
        \item Use of a hybrid container architecture is also known as hybrid containerization.
    \end{itemize}
\end{frame}

\subsection{Hybrid container architecture}\label{subsec:hybrid-container-architecture}
\begin{frame}{Hybrid container architecture}
    \begin{figure}[!t]
        \raggedright
        \begin{tikzpicture}[
            node/.style={
                scale=0.60,
                node distance = 2mm,
                rectangle,
                draw = black,
                thick,
                text centered,
                minimum height = 10mm,
                minimum width = 10mm
            },
            txt/.style = {
                scale = 0.60,
                node distance = 2mm,
                rectangle,
                text centered,
                minimum height = 2mm,
                minimum width = 10mm
            }
        ]
            \tikzstyle{layer} = [inner sep=1mm, draw=black, thick, fill= gray!30]
            \tikzstyle{vm} = [fill = gray!30]
            \tikzstyle{hw} = [fill = blue!35]
            \tikzstyle{sw} = [fill = blue!15]
            \tikzstyle{hv} = [fill = blue!5 ]

            %%%%%%%%%%%%%%%%%%%%%%%%%%%%%%%%%
            % Hybrid container architecture %
            %%%%%%%%%%%%%%%%%%%%%%%%%%%%%%%%%
            \node [node, hw, text width=10cm, yshift=-2mm, visible on=<1->] (HW) {Hardware};
            \node [node, hv, above = of HW, text width=10cm, yshift=-2mm, visible on=<2->] (HV1) {Hypervisor};
            \node [node, sw, above = of HV1, text width=10cm, yshift=-2mm, visible on=<3->] (OS1) {Host Operating System};
            % Virtual Machine 1
            \node [txt , vm, above = of OS1, text width=4.35cm, xshift = -2.60cm, visible on=<4->] (VM1) {Virtual Machine};
            \node [node, sw, above = of VM1, text width=4.45cm, yshift=-2mm, visible on=<5->] (OS2) {Guest OS};
            \node [node, hv, above = of OS2, text width=4.45cm, yshift=-2mm, visible on=<6->] (HV2) {Container Runtime};
            \node [node , sw, above = of HV2, text width=1.13cm, yshift=-2mm, xshift = -1.65cm, visible on=<7->] (CO1) {C1};
            \node [node , sw, right = of CO1, text width=1.13cm, xshift = -1mm, visible on=<8->] (CO2) {C2};
            \node [node , sw, right = of CO2, text width=1.13cm, xshift = -1mm, visible on=<9->] (CO3) {C3};
            \begin{scope}[on background layer]
                \node [fit = (VM1) (OS2) (HV2) (CO1) (CO2) (CO3), layer, visible on=<4->] {};
            \end{scope}
            % Virtual Machine 2
            \node [txt , right = of VM1, text width=4.45cm, xshift = 2mm, visible on=<10->] (VM2) {Virtual Machine};
            \node [node, sw, above = of VM2, text width=4.45cm, yshift=-2mm, visible on=<10->] (OS3) {Guest OS};
            \node [node, hv, above = of OS3, text width=4.45cm, yshift=-2mm, visible on=<10->] (HV3) {Container Runtime};
            \node [node , sw, above = of HV3, text width=1.13cm, yshift=-2mm, xshift = -1.65cm, visible on=<10->] (CO4) {C1};
            \node [node , sw, right = of CO4, text width=1.13cm, xshift = -1mm, visible on=<10->] (CO5) {C2};
            \node [node , sw, right = of CO5, text width=1.13cm, xshift = -1mm, visible on=<10->] (CO6) {C3};
            \begin{scope}[on background layer]
                \node [fit = (VM2) (OS3) (HV3) (CO4) (CO5) (CO6), layer, visible on=<10->] {};
            \end{scope}
        \end{tikzpicture}
        \caption{Hybrid container architecture}
    \end{figure}
\end{frame}

\subsection{Do windows have native containers?}\label{subsec:windows-containers}
\begin{frame}{Do windows have native containers?}
    \begin{itemize}[<+- | alert@+>]
        \item You can have native windows containers but not Linux native containers yet.
        \item Microsoft's native hypervisor solution is Hyper-V\@.
        \item Using Hyper-V Microsoft supports running VMs natively on Windows, for example, Ubuntu on Windows (WSL).
        \item Microsoft is working on the OS-level virtualization solution to run Linux native containers.
    \end{itemize}
\end{frame}
%----------------------------------------------------------------------------------------------------------------------%