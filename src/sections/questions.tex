%----------------------------------------------------------------------------------------------------------------------%


\section{Popular Questions}\label{sec:popular-questions}
%----------------------------------------------------------------------------------------------------------------------%

\subsection{Why Containers are more popular?}\label{subsec:containers-are-more-popular}
\begin{frame}{Why Containers are more popular?}
    \begin{itemize}[<+- | alert@+>]
        \item Resource isolation: predictable application performance.
        \item Resource utilization: high efficiency and density.
        \item Agile application creation and deployment.
        \item Continuous development, integration, and deployment.
        \item Loosely coupled, distributed, elastic, liberated micro-services.
        \item Environmental consistency across development, testing, and production.
    \end{itemize}
\end{frame}

\subsection{What is hybrid container architecture?}\label{subsec:hybrid-container-architecture?}
\begin{frame}{What is hybrid container architecture?}
    \begin{itemize}[<+- | alert@+>]
        \item A hybrid container architecture is an architecture combining virtualization on both hardware and OS levels.
        \item Example: The container engine and associated containers execute on top of a virtual machine.
        \item Use of a hybrid container architecture is also known as hybrid containerization.
        \item PICTURES GOES HERE
    \end{itemize}
\end{frame}


\subsection{Do windows have native containers?}\label{subsec:windows-containers}
\begin{frame}{Do windows have native containers?}
    \begin{itemize}[<+- | alert@+>]
        \item You can have windows native containers but not linux native containers yet.
        \item Microsoft native hypervisor solution is Hyper-V\@.
        \item Using Hyper-V Microsoft supports running VMs natively on Windows, example: ubuntu on windows (WSL).
        \item Microsoft is working on OS-level virtualization solution to run linux native containers.
    \end{itemize}
\end{frame}
%----------------------------------------------------------------------------------------------------------------------%