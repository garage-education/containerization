%----------------------------------------------------------------------------------------------------------------------%


\section{Virtualization}\label{sec:virtualization}
%----------------------------------------------------------------------------------------------------------------------%

\subsection{Virtual Machines}\label{subsec:virtual-machines}
\begin{frame}{Virtual Machines}
    \begin{itemize}
        \item Virtual machines (VMs) are an abstraction of physical hardware.
        \item It allows you to run multiple Virtual Machines (VMs) on a single physical server's CPU\@.
        \item Each VM includes a full copy of an operating system, the application, necessary binaries, and libraries - taking up tens of GBs.
        \item Virtualization allows applications to be isolated between VMs.
        \item Virtualization provides a level of security as the information of one application cannot be freely accessed by another application.
        \item Virtualization allows better utilization of resources in a physical server.
        \item Virtualization allows better scalability because an application can be added or updated easily,
        \item Virtualization reduces hardware costs, and much more.
        \item Virtualization can present a set of physical resources as a cluster of disposable virtual machines.
    \end{itemize}
\end{frame}

\subsection{Virtualized Deployment}\label{subsec:virtualized-deployment}
\begin{frame}{Virtualized Deployment}
    PICTURE GOES HERE
\end{frame}

\subsection{Hypervisor}\label{subsec:hypervisor}
\begin{frame}{Hypervisor}
    \begin{itemize}
        \item How virtualization is possible?
        \begin{itemize}
            \item A hypervisor is computer software, firmware or hardware that creates and runs virtual machines.
            \item The hypervisor allows multiple VMs to run on a single machine.
        \end{itemize}
        \item It is usually classified to 2 types:
        \begin{itemize}
            \item Type-1, native or bare-metal hypervisors .
            \item Examples: VMware ESX and Citrix Xen servers.
            \item Type-2 or hosted hypervisors
            \item Examples: VMware player and VirtualBox.
        \end{itemize}
        \item PICTURES GOES HERE
    \end{itemize}
\end{frame}
%----------------------------------------------------------------------------------------------------------------------%