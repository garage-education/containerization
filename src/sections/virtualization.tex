%----------------------------------------------------------------------------------------------------------------------%


\section{Virtualization}\label{sec:virtualization}
%----------------------------------------------------------------------------------------------------------------------%

\subsection{Virtual Machines}\label{subsec:virtual-machines}
\begin{frame}{Virtual Machines}
    \begin{itemize}[<+- | alert@+>]
        \item Virtual machines (VMs) are an abstraction of physical hardware.
        \item It allows you to run multiple Virtual Machines (VMs) on a single physical server's CPU\@.
        \item Each VM includes a full copy of an operating system, the application, necessary binaries, and libraries - taking up tens of GBs.
        \item Virtualization allows applications to be isolated between VMs.
        \item Virtualization provides a level of security as the information of one application cannot be freely accessed by another application.
        \item Virtualization allows better utilization of resources in a physical server.
        \item Virtualization allows better scalability because an application can be added or updated easily,
        \item Virtualization reduces hardware costs, and much more.
        \item Virtualization can present a set of physical resources as a cluster of disposable virtual machines.
    \end{itemize}
\end{frame}

\subsection{Virtualized Deployment}\label{subsec:virtualized-deployment}
\begin{frame}{Virtualized Deployment}
    \begin{tikzpicture}[
        node/.style={
            scale=0.75,
            node distance = 2mm,
            rectangle,
            draw = black,
            thick,
            text centered,
            minimum height = 10mm,
            minimum width = 10mm
        },
        txt/.style = {
            scale = 0.75,
            node distance = 2mm,
            rectangle,
            text centered,
            minimum height = 2mm,
            minimum width = 10mm
        }
    ]
        \tikzstyle{layer} = [inner sep=1mm, draw=black, thick, fill= gray!30]
        \tikzstyle{vm} = [fill = gray!30]
        \tikzstyle{hw} = [fill = blue!35]
        \tikzstyle{sw} = [fill = blue!15]
        \tikzstyle{hv} = [fill = blue!5 ]
        \tikzstyle{ar} = [fill = black  ]

        \node [node, sw, text width=1.95cm, visible on=<1->] (A11) {application};
        \node [node, sw, right = of A11, text width=1.95cm, visible on=<1->] (A12) {application};
        \node [node, sw, right = of A12, text width=1.95cm, visible on=<1->] (A13) {application};
        \node [node, hw, below = of A12, text width=7cm, visible on=<1->] (OS1) {Operating System};
        \node [node, hw, below = of OS1, text width=7cm, visible on=<1->] (HW1) {Hardware};

        \node [node, ar, right = of OS1, text width=1cm, visible on=<2->, single arrow] (AR1) {};

        \node [node, hw, right = of AR1, text width=7cm, visible on=<4->] (OS2) {Operating System};
        \node [node, hw, below = of OS2, text width=7cm, visible on=<3->] (HW2) {Hardware};
        \node [node, hv, above = of OS2, text width=7cm, visible on=<5->] (HV1) {Hypervisor};
        % Virtual Machine 2
        \node [txt , vm, above = of HV1, text width=3.0cm, xshift = -1.8cm, visible on=<6->] (VM1) {Virtual Machine};
        \node [node, hw, above = of VM1, text width=3.0cm, yshift=-2mm, visible on=<7->] (OS3) {OS};
        \node [node, hw, above = of OS3, text width=3.0cm, visible on=<8->] (BL1) {Bin/Lib};
        \node [node, sw, above = of BL1, text width=1.23cm, xshift = -0.9cm, visible on=<9->] (A21) {app};
        \node [node, sw, right = of A21, text width=1.23cm, visible on=<10->] (A22) {app};
        % Virtual Machine 2
        \node [txt , right = of VM1, text width=3.0cm, xshift = 1mm, visible on=<11->] (VM2) {Virtual Machine};
        \node [node, hw, above = of VM2, text width=3.0cm, yshift=-2mm, visible on=<11->] (OS4) {OS};
        \node [node, hw, above = of OS4, text width=3.0cm, visible on=<11->] (BL2) {Bin/Lib};
        \node [node, sw, above = of BL2, text width=1.23cm, xshift = -0.9cm, visible on=<11->] (A23) {app};
        \node [node, sw, right = of A23, text width=1.23cm, visible on=<11->] (A24) {app};

        \begin{scope}[on background layer]
            \node [fit = (VM1) (OS3) (BL1) (A21) (A22), layer, visible on=<6->] {};
        \end{scope}
        \begin{scope}[on background layer]
            \node [fit = (VM2) (OS4) (BL2) (A23) (A24), layer, visible on=<11->] {};
        \end{scope}
    \end{tikzpicture}
\end{frame}

\subsection{Hypervisor}\label{subsec:hypervisor}
\begin{frame}{Hypervisor}
    \begin{itemize}[<+- | alert@+>]
        \item How virtualization is possible?
        \begin{itemize}[<+- | alert@+>]
            \item A hypervisor is computer software, firmware or hardware that creates and runs virtual machines.
            \item The hypervisor allows multiple VMs to run on a single machine.
        \end{itemize}
        \item It is usually classified to 2 types:
        \begin{columns}[T,onlytextwidth]
            \column{0.5\textwidth}
            \begin{itemize}
                \item Type-1, native or bare-metal hypervisors.
                \item Type-2 or hosted hypervisors
            \end{itemize}
            \column{0.5\textwidth}
            \metroset{block=fill}
            \begin{block}{Examples}
                \begin{itemize}
                    \item Type-1: VMware ESX and Citrix Xen servers.
                    \item Type-2: VMware player and VirtualBox.
                \end{itemize}
            \end{block}
        \end{columns}
        \item PICTURES GOES HERE
    \end{itemize}
\end{frame}
%----------------------------------------------------------------------------------------------------------------------%