%----------------------------------------------------------------------------------------------------------------------%


\section{Containerization}\label{sec:containerization}
%----------------------------------------------------------------------------------------------------------------------%

\subsection{Containers}\label{subsec:containers}
\begin{frame}{What is a Container?}
    \begin{itemize}
        \item A container is an abstraction at the OS layer that packages code and dependencies together as a standardized unit of software.
        \item We can refer to container as container image since it is a self-contained piece of software that has everything in it needed to run.
        \item Containers take up less space than VMs, runs quickly and in isolation.
        \item A container image is a persisted, lightweight, standalone, executable package of software that includes everything needed to run an application.
        \item Containerization eliminates infrastructure wasted resources and utilizes them.
        \item Multiple containers can run on the same machine and share the OS kernel with other Containers, each running as isolated.
    \end{itemize}
\end{frame}

\subsection{Container Deployment}\label{subsec:container-deployment}
\begin{frame}{Container Deployment}
    PICTURE GOES HERE
\end{frame}

\subsection{OS-level virtualization}\label{subsec:os-level-virtualization}
\begin{frame}{OS-level virtualization}
    \begin{itemize}
        \item How containerization is possible?
        \begin{itemize}
            \item OS-level virtualization refers to an operating system paradigm in which the kernel allows the existence of multiple isolated user space instances.
            \item OS-level virtualization Solutions are the container engines.
            \item Container engine is a managed environment for deploying containerized applications.
        \end{itemize}
        \item User space instances have different names:
        \begin{itemize}
            \item Containers in \textbf{Docker} and Linux containers \textbf{LXC}.
            \item VPS in \textbf{OpenVZ}
            \item Virtual Kernel \textbf{DragonFly BSD}
        \end{itemize}
    \end{itemize}
\end{frame}
%----------------------------------------------------------------------------------------------------------------------%